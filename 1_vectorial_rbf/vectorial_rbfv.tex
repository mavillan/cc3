  \documentclass{beamer}

% Used packages
\usepackage[utf8]{inputenc}
\usepackage[spanish]{babel}
\usepackage{amssymb, amsmath, amsthm, amsfonts, esint, mathtools}




% There are many different themes available for Beamer. A comprehensive
% list with examples is given here:
% http://deic.uab.es/~iblanes/beamer_gallery/index_by_theme.html
% You can uncomment the themes below if you would like to use a different
% one:
%\usetheme{AnnArbor}
\usetheme{Antibes}
%\usetheme{Bergen}
%\usetheme{Berkeley}
%\usetheme{Berlin}
%\usetheme{Boadilla}
%\usetheme{boxes}
%\usetheme{CambridgeUS}
%\usetheme{Copenhagen}
%\usetheme{Darmstadt}
%\usetheme{default}
%\usetheme{Frankfurt}
%\usetheme{Goettingen}
%\usetheme{Hannover}
%\usetheme{Ilmenau}
%\usetheme{JuanLesPins}
%\usetheme{Luebeck}
%\usetheme{Madrid}
%\usetheme{Malmoe}
%\usetheme{Marburg}
%\usetheme{Montpellier}
%\usetheme{PaloAlto}
%\usetheme{Pittsburgh}
%\usetheme{Rochester}
%\usetheme{Singapore}
%\usetheme{Szeged}
%\usetheme{Warsaw}

\title{Interpolación de campos vectoriales sin divergencia}

% A subtitle is optional and this may be deleted
\subtitle{Aplicaciones de RBF}

\author{Martín Villanueva\inst{1}}
% - Give the names in the same order as the appear in the paper.
% - Use the \inst{?} command only if the authors have different
%   affiliation.

\institute[Universidad Técnica Federico Santa María] % (optional, but mostly needed)
{
  \inst{1}%
  Departamento de Informática\\
  Universidad Técnica Federico Santa María}
% - Use the \inst command only if there are several affiliations.
% - Keep it simple, no one is interested in your street address.

\date{Seminario de modelos y métodos cuantitativos, 2015}
% - Either use conference name or its abbreviation.
% - Not really informative to the audience, more for people (including
%   yourself) who are reading the slides online

%\subject{Theoretical Computer Science}
% This is only inserted into the PDF information catalog. Can be left
% out.

% If you have a file called "university-logo-filename.xxx", where xxx
% is a graphic format that can be processed by latex or pdflatex,
% resp., then you can add a logo as follows:

% \pgfdeclareimage[height=0.5cm]{university-logo}{university-logo-filename}
% \logo{\pgfuseimage{university-logo}}

% Delete this, if you do not want the table of contents to pop up at
% the beginning of each subsection:
\AtBeginSubsection[]
{
  \begin{frame}<beamer>{Outline}
    \tableofcontents[currentsection,currentsubsection]
  \end{frame}
}

% Let's get started
\begin{document}

\begin{frame}
  \titlepage
\end{frame}



\begin{frame}{Outline}
  \tableofcontents
  % You might wish to add the option [pausesections]
\end{frame}




% Section and subsections will appear in the presentation overview
% and table of contents.
\section{Preliminares}
\subsection{Campos vectoriales}

\begin{frame}{Definición de campo vectorial}
 
  \begin{definition}
    Un campo vectorial es una función $S:\mathbb{R}^n \rightarrow \mathbb{R}^m$, tal que
    $S(\mathbf{x})=(s_1(\mathbf{x}),\ldots,s_m(\mathbf{x}))$, donde $s_i: \mathbb{R}^n \rightarrow \mathbb{R}$
    son campos escalares.
  \end{definition}

  \begin{itemize}
  \item {
    Coloquialmente hablando, un campo vectorial $S:\mathbb{R}^n \rightarrow \mathbb{R}^m$, es una función
    que asigna a cada punto en $\mathbb{R}^n$, un único vector en el espacio $\mathbb{R}^m$
  }
  \end{itemize}
\end{frame}

\subsection{Divergencia}


\begin{frame}{Divergencia}
  \begin{definition}
    Sea $\mathbf{F}$ un campo vectorial, se define la divergencia en un punto $\mathbf{x_0}$ del espacio como
    $$
    \text{div }\mathbf{F}(\mathbf{x_0}) =
    \lim_{V \rightarrow \{\mathbf{x}_0\}} \frac{1}{V} \oiint_{\partial V} \mathbf{F} \cdot \mathbf{n} \ dS
    $$
    Donde $V$ es el volumen que contiene el punto, $\partial V$ es la supercie continua y cerrada que encierra
    tal volumen y $\mathbf{n}$ un vector unitario, normal (y saliente) a $\partial V$ en cada punto
  \end{definition}

  \begin{itemize}
  \item {
    La divergencia de un campo vectorial en un punto, representa la densidad volumétrica de flujo saliente a través de
    una superficie que encierra el volumen infinitesimal en el cual se encuentra tal punto.
  }
  \end{itemize}
\end{frame}

\begin{frame}{Divergencia}
 
  \begin{definition}
    Alternativamente, la divergencia de un campo $\mathbf{F}:\mathbb{R}^n \rightarrow \mathbb{R}^m$ se puede representar
    como:
    $$ \nabla \cdot \mathbf{F} =
    \nabla \cdot (F_1,\ \ldots \ ,F_m)=
    \frac{\partial F_1}{\partial x_1} + \ldots + \frac{\partial F_m}{\partial x_m} $$
  \end{definition}

  \begin{itemize}
  \item {
    El operador diferencial nabla se define $\nabla=(\frac{\partial}{\partial x_1}, \ \ldots \ , \frac{\partial}{\partial x_n})$
  }

  \end{itemize}
\end{frame}


\begin{frame}{Campos libres de divergencia}
  \begin{definition}
    Un campo $\mathbf{F}:\mathbb{R}^n \rightarrow \mathbb{R}^m$ se dice libre de divergencia, si para todo punto
    en $\mathbb{R}^n$ se cumple que:
    $$ \nabla \cdot \mathbf{F} = 0$$  
  \end{definition}

  \begin{itemize}
  \item {
    Intuitivamente, los campos libres de divergencia son aquellos en donde el flujo entrante en cada $\partial S$ que rodea un punto, es igual al flujo que sale.
  }
  \item {
    Se puede extender esta idea, enunciando que el flujo neto sobre una superficie cerrada en un campo libre de divergencia es cero (Teorema de Gauss).
  }
  \end{itemize}
\end{frame}


\begin{frame}{Linealidad}
  \begin{theorem}
  Sean $\mathbf{F},\mathbf{G}:\mathbb{R}^n \rightarrow \mathbb{R}^m$  dos campos vectoriales libres de divergencia, entonces
  su combinación lineal $\alpha \mathbf{F} + \beta \mathbf{G}$ (con $\alpha \text{ y } \beta$ escalares) también lo es.
  \end{theorem}

  \begin{itemize}
  \item {
    Esta propiedad de los campos sin divergencia, es fundamental para uno de los enfoques del problema que se plantea a continuación.
  }
  \end{itemize}
\end{frame}



\section{Problema planteado}

\begin{frame}{Definición del problema}
 
  \begin{block}{Problema}
    El problema consiste en dado un conjunto de datos dispersos $D=\{\mathbf{x}_j, \mathbf{d}_j \}_{j=1 \ldots N}$ con $\mathbf{x}_j=(x_j,y_j)$ y $\mathbf{d}_j = (d_{1j},d_{2j})$ (ambos en $\mathbb{R}^2$), encontrar un campo vectorial $\mathbf{S}$ \textit{libre de divergencia} tal que

    $$ \mathbf{S}(\mathbf{x}_j)=
    \begin{pmatrix} u(\mathbf{x}_j)\\v(\mathbf{x}_j)\end{pmatrix}=
    \begin{pmatrix} d_{1j} \\ d_{2j} \end{pmatrix}=
    \mathbf{d}_j
     $$
  \end{block}

  \begin{itemize}
  \item {
    Si no existiese la restricción de que $\mathbf{S}$ deba ser libre de divergencia, el problema se convierte simplemente
    en dividir los datos en $D_1=\{\mathbf{x}_j, d_{1j} \}_{j=1:N}$ y $D_2=\{\mathbf{x}_j, d_{2j} \}_{j=1:N}$ y
    determinar dos funciones escalares $u(\mathbf{x})$ y $v(\mathbf{x})$ que interpolan cada conjunto.
  }
  \end{itemize}
\end{frame}




\section{Soluciones Propuestas}

\subsection{Primer Enfoque}

\begin{frame}{Interpolar independientemente}
  \begin{itemize}
    \item{
      Esta solución consiste básicamente en hacer lo que se sugiere anteriormente; Particionar el conjunto inicial de datos $D$ en $D_1$ y $D_2$, y hallar dos funciones \textit{escalares} $u(\mathbf{x})$ y $v(\mathbf{x})$ que interpolan independientemente a cada conjunto respectivamente, \textbf{forzando} que se cumpla $u_x + u_y = 0$.
    }
    \item{
      \textbf{Problema:} Elegir dos bases funcionales (RBF) $\{B_k\}_{k=1:N}$ y ${\{\widetilde{B}_k\}_{k=1:N}}$ (no necesariamente la misma) tales que cualquier combinación lineal $u(\mathbf{x})=\sum \alpha_j B_j(\mathbf{x})$ y $v(\mathbf{x})=\sum \beta_j \widetilde{B}_j(\mathbf{x})$, cumplan con la restricción $u_x+v_y=0$ es un problema complejo.
    }
    \item{
      Resolver tal problema, implica determinar bases funcionales tales que
      $$\alpha_i \frac{\partial B_i}{\partial x} + \beta_j \frac{\partial \widetilde{B}_j}{\partial y} = 0 $$
    }
  \end{itemize}
\end{frame}

\subsection{Segundo Enfoque}

\begin{frame}{Suma de campos libres de divergencia}

\begin{block}{Solución propuesta}
  Construir el campo vectorial $\mathbf{S}$ como una combinación lineal de funciones matriciales radiales  
  $\mathbf{\Phi}_k(\mathbf{x})=\widetilde{\mathbf{\Phi}_k}(\mathbf{||\mathbf{x}-\mathbf{x}_k||})$ de la siguiente manera

  $$ \mathbf{S}(\mathbf{x}) =
  \begin{pmatrix} u(\mathbf{x})\\v(\mathbf{x})\end{pmatrix} =
  \sum_{j=1}^{N} \mathbf{\Phi}_j(\mathbf{x}) \cdot \mathbf{c}_j
  $$

  donde $\mathbf{c}_j=\begin{pmatrix} c_{1j} \\ c_{2j} \end{pmatrix}$

\end{block}
\end{frame}

\begin{frame}
  \begin{block}{Solución propuesta (continuación)}
    $\mathbf{\Phi}_j(\mathbf{x})$ se construye como sigue:

    $$ \mathbf{\Phi}_j(\mathbf{x}) =
    (\nabla \nabla^T - \nabla^2 I)\psi_j(\mathbf{x})=
    \begin{pmatrix}
    -\frac{\partial^2}{\partial y^2} & \frac{\partial^2}{\partial x \partial y} \\
    \frac{\partial^2}{\partial y \partial x} & -\frac{\partial^2}{\partial x^2}
    \end{pmatrix}\psi_j(\mathbf{x}) $$
  \end{block}

  y $\psi_j(\mathbf{x})$ es una RBF centrada en el punto $\mathbf{x}_j
  $
\end{frame}


\begin{frame}{Suma de campos libres de divergencia}
\begin{itemize}
  \item A continuación se pasa a detallar, porqué la solución propuesta funciona
\end{itemize}

\begin{block}{Explicación}
  Considerando la construcción anterior de $\mathbf{S}$, se puede reescribir como

  $$
  \mathbf{S}(\mathbf{x})=\sum_{j=1}^N
  \begin{pmatrix}
  \phi_{11_{j}} & \phi_{12_{j}} \\
  \phi_{21_{j}} & \phi_{22_{j}}
  \end{pmatrix} \cdot
  \begin{pmatrix} c_{1j} \\ c_{2j} \end{pmatrix} =
  \sum_{j=1}^N \Big(
  c_{1j} \begin{pmatrix} \phi_{11_{j}} \\ \phi_{21_{j}} \end{pmatrix} +
  c_{2j} \begin{pmatrix} \phi_{12_{j}} \\ \phi_{22_{j}} \end{pmatrix} \Big)
  $$  
\end{block}
 
\end{frame}

\begin{frame}{Suma de campos libres de divergencia}
  \begin{block}{Explicación (continuación)}
  donde:
  $$
  \mathbf{L}_j(\mathbf{x})=\begin{pmatrix} \phi_{11_{j}} \\ \phi_{21_{j}} \end{pmatrix} =
  \begin{pmatrix} -\frac{\partial^2}{\partial y^2}\psi_{j}(\mathbf{x}) \\
  \frac{\partial^2}{\partial y \partial x}  \psi_{j}(\mathbf{x}) \end{pmatrix}
  $$
  $$
  \mathbf{R}_j(\mathbf{x})=\begin{pmatrix} \phi_{12_{j}} \\ \phi_{22_{j}} \end{pmatrix} =
  \begin{pmatrix} \frac{\partial^2}{\partial x \partial y}  \psi_{j}(\mathbf{x})  \\
  -\frac{\partial^2}{\partial x^2}\psi_{j}(\mathbf{x})
   \end{pmatrix}
  $$
  Cada uno de estos vectores columna son en sí campos vectoriales y cumplen lo siguiente:
  $$
  \nabla \cdot \mathbf{L}_j =
  -\frac{\partial^3}{\partial x \partial y^2} \psi_{j}(\mathbf{x}) + \frac{\partial^3}{\partial y^2 \partial x} \psi_{j}(\mathbf{x})
  $$

  \end{block}
\end{frame}

\begin{frame}{Suma de campos libres de divergencia}
  \begin{block}{Explicación (continuación)}
  $$
  \nabla \cdot \mathbf{R}_j =
  \frac{\partial^3}{\partial x^2 \partial y} \psi_{j}(\mathbf{x}) -\frac{\partial^3}{\partial y \partial x^2} \psi_{j}(\mathbf{x})
  $$
  luego si las funciones RBF $\psi_{j}(\mathbf{x})$ tienen segundas derivadas parciales continuas, por el teorema de Schwarz, no importa
  el orden en que se evalúan las derivadas, y por lo tanto la divergencia de ambos campos es analíticamente $0$. Como
  $\mathbf{S}(\mathbf{x})$ no es más que un combinación lineal de campos sin divergencia, entonces por construcción, también debe serlo.
  \end{block}

  \begin{itemize}
    \item La mayoría de las RBF ocupadas tienen segundas derivadas parciales continuas!.
  \end{itemize}
\end{frame}

\begin{frame}
  \begin{example}
    Veamos en la siguiente demostración que tal propiedad se cumple para algunas de las RBF más utilizadas...
  \end{example}


\end{frame}

\begin{frame}{Planteamiento del sistema lineal}
  \begin{itemize}
  \item Al igual que con interpolación con RBF para campos escalares, se debe evaluar $\mathbf{S}(\mathbf{x})$ para los $N$ datos
  y construir un sistema lineal a resolver.
  \item Pese a que el campo vectorial de interpolación posee una forma distinta a la función de interpolación con RBF ``clásico", es posible
  escribir la evaluación del campo interpolador para cada punto $\mathbf{x}_j$ de forma matricial, como se muestra a continuación.
  \item \textbf{Notación:} $\phi_{11}^{\mathbf{x}_k}(\mathbf{x}_j)$ es la evaluación de la función $\phi_{11}$ que viene de la matriz
  $\mathbf{\Phi}_{k}$ (centrada en el punto $\mathbf{x}_k$) y evaluada en el punto $\mathbf{x}_j$.
\end{itemize}  
\end{frame}


\begin{frame}{Planteamiento del sistema lineal}

  \begin{block}{Matriz de interpolación}
  $$
  \begin{pmatrix}
  \phi_{11}^{\mathbf{x}_1}(\mathbf{x}_1) & \ldots & \phi_{11}^{\mathbf{x}_N}(\mathbf{x}_1) &
  \phi_{12}^{\mathbf{x}_1}(\mathbf{x}_1) & \ldots & \phi_{12}^{\mathbf{x}_N}(\mathbf{x}_1) \\
  \vdots & \vdots & \vdots & \vdots & \vdots & \vdots \\
  \phi_{11}^{\mathbf{x}_1}(\mathbf{x}_N) & \ldots & \phi_{11}^{\mathbf{x}_N}(\mathbf{x}_N) &
  \phi_{12}^{\mathbf{x}_1}(\mathbf{x}_N) & \ldots & \phi_{12}^{\mathbf{x}_N}(\mathbf{x}_N) \\
  \phi_{21}^{\mathbf{x}_1}(\mathbf{x}_1) & \ldots & \phi_{21}^{\mathbf{x}_N}(\mathbf{x}_1) &
  \phi_{22}^{\mathbf{x}_1}(\mathbf{x}_1) & \ldots & \phi_{22}^{\mathbf{x}_N}(\mathbf{x}_1) \\
  \vdots & \vdots & \vdots & \vdots & \vdots & \vdots \\
  \phi_{21}^{\mathbf{x}_1}(\mathbf{x}_N) & \ldots & \phi_{21}^{\mathbf{x}_N}(\mathbf{x}_N) &
  \phi_{22}^{\mathbf{x}_1}(\mathbf{x}_N) & \ldots & \phi_{22}^{\mathbf{x}_N}(\mathbf{x}_N) \\
  \end{pmatrix} \cdot
  \begin{pmatrix}
    c_{11} \\ \vdots \\ c_{1N} \\ c_{21} \\ \vdots \\ c_{2N}
  \end{pmatrix} =
  \begin{pmatrix}
    d_{11} \\ \vdots \\ d_{1N} \\ d_{21} \\ \vdots \\ d_{2N}   
  \end{pmatrix}
  $$
  \end{block}
\end{frame}

\begin{frame}{Planteamiento del sistema lineal}
\begin{itemize}
  \item La matriz de interpolación $A$ esta compuesta de las siguientes $4$ submatrices  
\end{itemize}

\begin{block}{Composición de la matriz de interpolación}
  \begin{itemize}
    \item $A_{11_{(i,j)}} = \phi_{11}^{\mathbf{x}_j}(\mathbf{x}_i)$ para $i=1:N$, $j:1:N$
    \item $A_{12_{(i,j)}} = \phi_{12}^{\mathbf{x}_j}(\mathbf{x}_i)$ para $i=1:N$, $j:N+1:2N$
    \item $A_{21_{(i,j)}} = \phi_{21}^{\mathbf{x}_j}(\mathbf{x}_i)$ para $i=N+1:2N$, $j:1:N$
    \item $A_{22_{(i,j)}} = \phi_{22}^{\mathbf{x}_j}(\mathbf{x}_i)$ para $i=N+1:2N$, $j:N+1:2N$
  \end{itemize}
\end{block}
 
\end{frame}


% Placing a * after \section means it will not show in the
% outline or table of contents.
\section{Experimentación}

\begin{frame}{Experimentación}
 
\begin{example}
  Veamos en la siguiente demostración la implementación de este método y los resultados que obtiene...
\end{example}
\end{frame}




% All of the following is optional and typically not needed.
% \appendix
% \section<presentation>*{\appendixname}
% \subsection<presentation>*{For Further Reading}

% \begin{frame}[allowframebreaks]
%   \frametitle<presentation>{For Further Reading}
    
%   \begin{thebibliography}{10}
    
%   \beamertemplatebookbibitems
%   % Start with overview books.

%   \bibitem{Author1990}
%     A.~Author.
%     \newblock {\em Handbook of Everything}.
%     \newblock Some Press, 1990.
 
    
%   \beamertemplatearticlebibitems
%   % Followed by interesting articles. Keep the list short.

%   \bibitem{Someone2000}
%     S.~Someone.
%     \newblock On this and that.
%     \newblock {\em Journal of This and That}, 2(1):50--100,
%     2000.
%   \end{thebibliography}
% \end{frame}

\end{document}